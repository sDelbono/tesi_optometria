



Gli studi condotti negli ultimi anni hanno posto l'attenzione su come il nostro corpo sia funzionalmente collegato e le sue parti si influenzino reciprocamente. Questa relazione funzionale non si attua tra singoli elementi, come occhio e lingua, ma tra sistemi, come visivo e stomatognatico. Poichè sono inclusi in un sistema più complesso, quale è il nostro corpo, esso si adatta ai cambiamenti in modo diverso per ognuno di noi.
\\\
Il corpo recepisce le informazioni su come organizzarsi nello spazio da stimoli provenienti dal mondo esterno. Attraverso questo meccanismo stimolo-risposta si sviluppa la postura corporea. Dal punto di vista motorio il corpo si adatta agli stimoli esterni per sopravvivere e svolgere le sue attività, questo porta ad assumere quindi la postura più consona alla situazione e alle proprie esigenze. 
\\\
La funzione di informare i sistemi è affidata al Sistema-Tonico-Posturale \emph{S.T.P.} costituito da un insieme di strutture comunicanti. L'S.T.P. necessita di un input che proviene dagli estrocettori e propriocettori (esempio: occhio,bocca, piede, pelle, muscoli), il quale viene rielaborato e trasformato in output che traduce in gesto motorio il segnale.
\\\
L'occhio fornisce al cervello la maggior parte delle informazioni che servono per interpretare lo spazio che ci circonda, quindi una disfunzione del sistema visivo può provocare degli squilibri nei vari sistemi e viceversa. In particolare in vari studi è stato evidenziato come ci sia un collegamento tra le malocclusioni e i difetti di convergenza. Sembra infatti che disfunzioni dell'apparato stomatognatico influenzino le funzionalità del sistema visivo, portando ad adattamenti della postura del capo\footnote{Armando Silvestrini-Biavati, Marco Migliorati, Eleonora Demarziani, Simona Tecco, Piero Silvestrini-Biavati, Antonella Polimeni and Matteo Saccucci \emph{Clinical association between teeth malocclusions, wrong posture and ocular convergence disorders: an epidemiological investigation on primary school children}}\footnote{Antonino Marco Cuccia e Carola Caradonna \emph{Binocular motility system and temporomandibular joint internal derangement: A study in adults}}. 


È in questo contesto che si colloca questo studio, basato sull'assunzione che dalla disfunzione dell'apparato stomatognatico si ottiene l'adattamento da parte degli altri sistemi, in particolare del sistema visivo. Per ottenere i risultati si vuole osservare se e come cambia il sistema visivo in soggetti con disfunzioni generali primarie dell'apparato stomatognatico (masticazione, deglutizione, respirazione) dopo aver seguito un trattamento logopedico e osteopatico. Il trattamento prevede l'utilizzo della Terapia Miofunzionale, cioè un percorso terapeutico volto ad insegnare la corretta posizione della lingua sia durante il riposo, che durante la deglutizione per portare al riequilibrio della funzione della muscolatura oro-facciale. La terapia viene svolta con l'insegnamento di esercizi volti al rilassamento, allungamento e rinforzo dei muscoli coinvolti nelle funzioni orali.  Viene poi integrata con delicate manipolazioni eseguite prevalentemente nei distretti di cranio, collo e torace con l'obiettivo di rilasciare le tensioni muscolari presenti.
\\\
Lo svolgimento della ricerca è avvenuto effettuando una valutazione optometrica pre e post trattamento, osservando diverse componenti del sistema visivo, tra cui il difetto refrattivo, la stabilità, l'oculomotricità, la convergenza e l'accomodazione.
L'obiettivo è verificare che l'occhio sia correlato al resto del corpo e in particolare all'apparato stomatognatico, attraverso strutture che sono in continuo adattamento per ricercare sempre l'equilibrio complessivo e comprendere quali sono le caratteristiche visive maggiormente collegate. 
\\\
In questa ricerca si è rivelato fondamentale il confronto con vari professionisti, inquadrando il lavoro in un approccio multidisciplinare.


I contenuti della ricerca saranno divisi in 3 capitoli. Nel primo capitolo verranno esposte le caratteristiche del sistema visivo dal punto di vista anatomico e fisiologico, con annessa la descrizione delle funzionalità analizzate durante la ricerca. Nel secondo si porrà l'attenzione sulla postura e sull'apparato stomatognatico, descrivendo i collegamenti anatomici e fisiologici che lo legano al sistema visivo. Nel terzo capitolo verranno descritti lo svolgimento e i risultati ottenuti dallo studio effettuato, concludendo con le considerazioni e proponendo eventuali sviluppi futuri.
