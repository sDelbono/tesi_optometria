\documentclass [a4paper,11pt]{book}
\usepackage[a4paper,plainpages=false,pdfpagelabels]{hyperref}
\usepackage[italian]{babel}
\usepackage[latin1]{inputenc}
\usepackage[font=small,labelfont=bf]{caption}

%%%%%%%%%
\usepackage{cite}
\usepackage{color}
\usepackage{units}
\usepackage{bold-extra}
\usepackage{array}
\usepackage{calc}
\newcolumntype{N}{@{}m{0pt}@{}}

\usepackage[framemethod=tikz]{mdframed}
\global\mdfdefinestyle{defStyle}{%
leftmargin=1cm,%
rightmargin=1cm,%
roundcorner=1,%
frametitlealignment=\center
}

\usepackage{amsmath}
\usepackage{listings}
\usepackage{cite}
\usepackage{url} 
\usepackage{graphicx}
\usepackage{fancyhdr}
\usepackage[italian]{babel}
\usepackage{fixltx2e}
\usepackage[labelfont=bf]{caption}
\usepackage{float}
\usepackage[utf8]{inputenc}
\usepackage{geometry}
\graphicspath{{images/}}
\captionsetup{labelfont=bf}
\pagestyle{fancy}
\fancyhf{} %
\fancyhead[LE]{\bfseries\leftmark}
\fancyhead[RO]{\bfseries\rightmark}
\fancyfoot[C]{\thepage}

\def\signed #1{{\leavevmode\unskip\nobreak\hfil\penalty50\hskip2em
  \hbox{}\nobreak\hfil(#1)%
  \parfillskip=0pt \finalhyphendemerits=0 \endgraf}}

\newsavebox\mybox
\newenvironment{aquote}[1]
  {\savebox\mybox{#1}\begin{quote}}
  {\signed{\usebox\mybox}\end{quote}}
%%%%%%%%
\usepackage{setspace}
\linespread{1.2}


\begin{document}

%%%%%%%%
\frontmatter

% Create a new page
\newpage
% Set borders
\newgeometry{top=3cm,bottom=3cm,inner=2.7cm,outer=2.7cm}
% Remove page number
\thispagestyle{empty}
% Remove default indentation of paragraph
\noindent
% Move the content a bit on the left
\hspace*{-1mm}
% Include the logo
\includegraphics[width=0.2\columnwidth]{source/immagini/logo-bicocca-gray.jpg}
% Place some space between text and image
\hspace*{2mm}
% Open a box in which place the header
% use \par to space the lines
\begin{minipage}[b][][c]{0.8\columnwidth}
\begin{spacing}{1.4}
{\large\textsc{Universit\'a degli Studi di Milano Bicocca}\par}
{\large\textbf{Scuola di Scienze}\par}
{\large\textbf{Dipartimento di Scienze dei Materiali}\par}
{\large\textbf{Corso di Laurea in Ottica e Optometria}\par}
\end{spacing}
\end{minipage}
% Vertical fill space
\vfill


\begin{center}
{\Huge\textsc{\textbf{Correlazione tra sistema visivo e apparato stomatognatico }}\par}
\end{center}
% Fill the remaining space
\vfill
\large
% Place the advisor under title, on the left
\begin{flushleft}
\textbf{Relatore:}\\
\hspace*{2em}\textit{Prof.} Maurizio \textsc{Acciarri}
\end{flushleft}
% Place co-advisors under advisor
\begin{flushleft}
\textbf{Co-relatore:}\\
\hspace*{2em}\textit{Prof.ssa} Nadia \textsc{Mattioli}\\
\end{flushleft}
% Leave some space
\vskip 1cm
% Write candidate on the right
\begin{flushright}
\textbf{Relazione della prova finale di:}\\
\hspace*{2em}Serena \textsc{Delbono}\\
\hspace*{2em}Matricola 782525

\textbf{ \\ Appello di laurea:}\\
\hspace*{2em}16 Marzo 2017
\end{flushright}
\vfill
% Place year at bottom
\begin{center}
\textsc{\textbf{Anno Accademico 2015-2016}}
\end{center}
\restoregeometry


%%%%%%%%%%%%%%%%%%%


\mainmatter
\chapter*{Introduzione}
\addcontentsline{toc}{chapter}{Introduzione}
Gli studi condotti negli ultimi anni hanno posto l'attenzione su come il nostro corpo sia funzionalmente collegato e le sue parti si influenzino reciprocamente. Questa relazione funzionale non si attua tra singoli elementi, come occhio e lingua, ma tra sistemi, come visivo e stomatognatico. Poich� sono inclusi in un sistema pi� complesso, quale � il nostro corpo, esso si adatta ai cambiamenti in modo diverso per ognuno di noi\footnote{\emph{Occlusione vs oculomotricit�: stato dell'arte} di Giorgetti, Deodato, Malpassi}.
\\\
Il corpo recepisce le informazioni su come organizzarsi nello spazio da stimoli provenienti dal mondo esterno. Attraverso questo meccanismo stimolo-risposta si sviluppa la postura corporea. Dal punto di vista motorio il corpo si adatta agli stimoli esterni per sopravvivere e svolgere le sue attivit�, questo porta ad assumere quindi la postura pi� consona alla situazione e alle proprie esigenze. 
\\\
La funzione di informare i sistemi � affidata al Sistema-Tonico-Posturale \emph{S.T.P.} costituito da un insieme di strutture comunicanti. L'S.T.P. necessita di un input che proviene dagli estrocettori e propriocettori (esempio: occhio,bocca, piede, pelle, muscoli), il quale viene rielaborato e trasformato in output che traduce in gesto motorio il segnale\footnote{\url{http://www.associazioneitalianastudioericercaposturologia.it/allegati/Articolo\%20AIP\%20Salvatore\%20Mautone.pdf}}.
\\\
L'occhio fornisce al cervello la maggior parte delle informazioni che servono per interpretare lo spazio che ci circonda, quindi una disfunzione del sistema visivo pu� provocare degli squilibri nei vari sistemi e viceversa. In particolare in vari studi � stato evidenziato come ci sia un collegamento tra le malocclusioni e i difetti di convergenza. Sembra infatti che disfunzioni dell'apparato stomatognatico influenzino le funzionalit� del sistema visivo, portando ad adattamenti della postura del capo\footnote{Armando Silvestrini-Biavati, Marco Migliorati, Eleonora Demarziani, Simona Tecco, Piero Silvestrini-Biavati, Antonella Polimeni and Matteo Saccucci \emph{Clinical association between teeth malocclusions, wrong posture and ocular convergence disorders: an epidemiological investigation on primary school children}}\footnote{Antonino Marco Cuccia e Carola Caradonna \emph{Binocular motility system and temporomandibular joint internal derangement: A study in adults}}. 


� in questo contesto che si colloca questo studio, basato sull'assunzione che dalla disfunzione dell'apparato stomatognatico si ottiene l'adattamento da parte degli altri sistemi, in particolare del sistema visivo. Per ottenere i risultati si vuole osservare se e come cambia il sistema visivo in soggetti con disfunzioni generali primarie dell'apparato stomatognatico (masticazione, deglutizione, respirazione) dopo aver seguito un trattamento logopedico e osteopatico. Il trattamento prevede l'utilizzo della Terapia Miofunzionale, cio� un percorso terapeutico volto ad insegnare la corretta posizione della lingua sia durante il riposo, che durante la deglutizione per portare al riequilibrio della funzione della muscolatura oro-facciale. La terapia viene svolta con l'insegnamento di esercizi volti al rilassamento, allungamento e rinforzo dei muscoli coinvolti nelle funzioni orali.  Viene poi integrata con delicate manipolazioni eseguite prevalentemente nei distretti di cranio, collo e torace con l'obiettivo di rilasciare le tensioni muscolari presenti.
\\\
Lo svolgimento della ricerca � avvenuto effettuando una valutazione optometrica pre e post trattamento, osservando diverse componenti del sistema visivo, tra cui il difetto refrattivo, la stabilit�, l'oculomotricit�, la convergenza e l'accomodazione.
L'obiettivo � verificare che l'occhio sia correlato al resto del corpo e in particolare all'apparato stomatognatico, attraverso strutture che sono in continuo adattamento per ricercare sempre l'equilibrio complessivo e comprendere quali sono le caratteristiche visive maggiormente collegate. 
\\\
In questa ricerca si � rivelato fondamentale il confronto con vari professionisti, inquadrando il lavoro in un approccio multidisciplinare.


I contenuti della ricerca saranno divisi in 3 capitoli. Nel primo capitolo verranno esposte le caratteristiche del sistema visivo dal punto di vista anatomico e fisiologico, con annessa la descrizione delle funzionalit� analizzate durante la ricerca. Nel secondo si porr� l'attenzione sulla postura e sull'apparato stomatognatico, descrivendo i collegamenti anatomici e fisiologici che lo legano al sistema visivo. Nel terzo capitolo verranno descritti lo svolgimento e i risultati ottenuti dallo studio effettuato, concludendo con le considerazioni e proponendo eventuali sviluppi futuri.






\end{document}