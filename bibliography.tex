\begin{thebibliography}{9}
\addcontentsline{toc}{chapter}{Bibliografia}

\bibitem{bib1}
Giorgetti, Deodato, Malpassi: \emph{Occlusione vs oculomotricità: stato dell’arte}

\bibitem{bib2}
Angelo Bairati: \emph{Trattato di anatomia umana – sistema nervoso periferico, organi di senso, apparato tegumentario}, II edizione, Volume III, Edizioni Minerva Medica

\bibitem{bib3}
G.C. Balboni, A. Bastianini, E. Brizzi, L. Comparini, G. Filogamo, G. Giordano-Lanza, C. E. Grossi, F. A. Manzoli, G. Marinozzi, P. Motta, G. E. Orlandini, A. Passaponti, E. Reale, A. Ruggeri, A. Santoro, D. Zaccheo \emph{Anatomia Umana 3}, edizioni edi.ermes

\bibitem{bib4}
William J.: \emph{Butterworth Heinemann Elsevier “Borish’s Clinical Refraction} Benjamin ed.

\bibitem{bib5}
A. Cuccia, C. Caradonna: \emph{The relationship between the stomatognathic system and body posture}, 2009

\bibitem{bib6}
Philippe Caiazzo: \emph{TOP terapia osteopatico-posturale}, edizioni Marrapese, editore-Roma

\bibitem{bib7}
R. Ciancaglini, R. Gelmetti, E. Lazzari: \emph{Evoluzione degli studi sulla relazione tra occlusione e postura}, da Mondo Ortodontico gennaio, 2008

\bibitem{bib8}
Rhoda P. Erhardt, \emph{Developmental Visual Dysfunction: Models for Assessment and Management}, capitolo 3, Erhardt Developmental Products 1990.

\bibitem{bib9}
Mautone Salvatore: \emph{“Sistema Tonico posturale: il direttore d’orchestra!”}

\bibitem{bib10}
David B. Elliott: \emph{Clinical procedures in primary eye care}, Elsevier Saunders Fourth Edition, 2014

\bibitem{bib11}
Bruno Garuffo: Materiale Didattico del corso Tecniche Fisiche per l’Optometria Generale II modulo, 2014-2015

\bibitem{bib12}
Marzia Lecchi: Materiale Didattico del corso Fisiologia Generale ed Oculare II modulo, 2013-2014.

\bibitem{bib13}
Mauro Faini: \emph{Lezioni di optometria}

\bibitem{bib14}
Corinna Galbero: \emph{Influenza della dominanza oculare sull’equilibrio posturale in soggetti giovani}, tesi di laurea in Ottica e Optometria dell’Università di Padova 

\bibitem{bib15}
Alice Delbono: \emph{Correlazione tra squilibrio muscolare oro-facciale e disfunzioni di interesse osteopatico: studio su 21 soggetti}, tesi di laurea in Logopedia, 2015

\bibitem{bib16}
Maria Cristina Fresa: \emph{Deglutizione fisiologica, deglutizione disfunzionale, significato neurofisiologico della deglutizione e sua interferenza sulla postura}, tesi di laurea in Medicina e Chirurgia dell’Università di Pisa, specializzazione in Medicina fisica e riabilitativa, 2011

\bibitem{bib17}
http://www.gruppocdc.it/pazienti/prestazioni/laboratorio-analisi-cliniche/chimica-clinica-tossicologica/7-cdc/cdc-ita/314-trattamento-dello-squilibrio-muscolare-oro-facciale

\bibitem{bib18}
https://alessandrogarlinzoni.wordpress.com/2016/10/28/perche-locchio-influenza-la-postura/

\bibitem{bib19}
Wikipedia, www.wikipedia.org

\end{thebibliography} 