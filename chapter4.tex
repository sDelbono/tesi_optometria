\chapter{Conclusioni e Sviluppi Futuri}

A conclusione della ricerca e dell’analisi statistica dei dati si osserva che, secondo il test di Wilcoxon, non esiste la correlazione tra molte funzioni del sistema visivo e l’apparato stomatognatico e che il cambiamento osservato sia da ritenersi casuale e non dovuto al trattamento logopedico-osteopatico.
 
Analizzando i valori medi è stato possibile riscontrare un riduzione dei valori nei test di Punto Prossimo di Convergenza (diritto e in basso), Cover test (a destra e in alto a destra) e Stereopsi vicino. Tra questi è stato osservato un cambiamento da considerarsi statisticamente significativo solo nel test della Stereopsi vicino. Il sistema visivo ha infatti trovato una migliore stabilità in seguito al trattamento all’apparato stomatognatico, permettendo una maggiore consapevolezza dello spazio e quindi un miglioramento della percezione della profondità. L’aumento della stereopsi può essere legato al perfezionamento della binocularità del soggetto e quindi alle altre funzioni in cui è presente il cambiamento come la convergenza e la stabilità dei muscoli oculari che influenzano le forie.
\\\

Per le funzioni analizzate è possibile che il cambiamento sia casuale, ma, dal confronto dei grafici del gruppo sperimentale e del gruppo di controllo, dal risultato ottenuto per la stereopsi e dal cambiamento dei singoli soggetti, è possibile ancora ipotizzare l’efficacia del trattamento logopedico-osteopatico effettuato su soggetti con primarietà linguale. Ciò che può essere affermato ora è che la rieducazione delle funzioni dell’apparato stomatognatico ha permesso al sistema visivo di trovare una nuova stabilità aumentando la percezione dello spazio.
\\\ \\\ \\\ \\\
In questa ricerca si è voluto effettuare l’analisi statistica considerando i soggetti come un gruppo, piuttosto che come singoli, poiché l’ampiezza del campione è stata ritenuta abbastanza grande per effettuare l’analisi statistica di Wilcoxon, \emph{ma}, sebbene l’approccio di  Wilcoxon  abbia confutato la tesi, i risultati mostrano comunque un miglioramento che potrebbe essere reso visibile utilizzando un altro approccio. Perciò questo lavoro si pone come un un punto di partenza per nuovi tipi di analisi in modo da poter confermare o smentire i risulatati ottenuti con Wilcoxon.
\\\ \\\ \\\ \\\
\section{Sviluppi futuri}

Alla luce delle conclusioni sopra descritte si possono pensare a diversi sviluppi che continuino il lavoro già svolto. A questo scopo sono state pensate diverse alternative:
\begin{enumerate}
\item effettuare una statistica a caso singolo sui soggetti che hanno evidenziato il miglioramento, per poter smentire o confermare che il trattamento effettuato non è risultato efficace sul sistema visivo.
\item approfondire il cambiamento del punto prossimo di convergenza utilizzando un campione più ampio per comprendere quanto la disfunzione della deglutizione possa influire sulla convergenza. Si potrebbe, infatti, effettuare un confronto con quanto riportato in letteratura, dove si registra un cambiamento della convergenza agendo sulle mal occlusioni dentali.
\item valutare quanto il solo risultato del cover test possa essere influenzato dalla disfunzione stomatognatica e comprendere quanto questo test possa essere affidabile nel valutare la funzione della muscolatura estrinseca.
\item rieffettuare l’analisi di Wilcoxon con un campione più ampio, possibilmente concentrandosi sui test in cui questo studio ha riportato un miglioramento.
\end{enumerate}
\\\ \\\ \\\
Inoltre, affinché l’analisi sia affidabile il più possibile:
\begin{itemize}
\item si potrebbe ridurre la quantità di test da sottoporre, oppure utilizzare un diverso metodo. Ad esempio il cover test nelle 9 posizioni di sguardo si è rivelato molto stancante per i soggetti, perciò si potrebbe utilizzare un test più immediato come la Bernell Muscle Imbalance Measure sempre proposto nelle 9 posizioni di sguardo.
\item sarebbe preferibile sottoporre i soggetti ad un questionario riguardo la loro condizione di stanchezza prima di ogni valutazione optometrica, per assicurarsi che le valutazioni siano effettuate sotto le stesse condizioni.
\item supervisionare maggiormente il soggetto nell’esecuzione degli esercizi. Nella ricerca effettuata c’era una forte dipendenza dall’impegno quotidiano dei soggetti, i quali sono stati seguiti dalla laureanda e dalla logopedista attraverso la compilazione di un diario a casa e attraverso appuntamenti in ambulatorio.
\item usare un’alternativa al Cover test per indagare le funzioni muscolari. Essendo un test che valuta qualitativamente la binocularità del soggetto, non è possibile determinare se il risultato ottenuto sia dovuto a disfunzioni muscolari o a uno squilibrio della visione binoculare. Esso è stato utilizzato perché si è ritenuto che per aver una binucularità perfetta sarebbe necessario possedere una corretta percezione dello spazio integrata con un buona stabilità muscolare e il cover test può fornire dati precisi e fini, quindi confrontabili, dal momento che indaga la presenza di forie\cite{bib15}. Però l’alterata percezione spaziale non è una variabile controllabile durante lo svolgimento del cover test, per questo si potrebbe tentare di utilizzare un test come il DEM o NSUCO, che permettono una valutazione quantitativa, anche se non precisa, delle abilità dei muscoli oculari e valutare se queste rimangono le stesse o migliorano.
\end{itemize}
