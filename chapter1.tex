
\chapter{Sistema visivo}
\\\
L’apparato visivo è il più complesso degli apparati dei sensi ed ha la massima importanza perché è quello che più di altri ci permette di conoscere il mondo esterno. Questo avviene perché l’apparato non provvede solo alla percezione qualitativa e quantitativa della luce, ma trasmette le informazioni al cervello in segnali topograficamente organizzati in modo tale che la recezione si traduce in immagini sufficientemente fedeli di ciò che stiamo osservando. Esso permette quindi lo sviluppo della visione, in quanto capacità di interpretare e capire ciò che vediamo.

Gli elementi anatomici che compongono l’apparato visivo sono: bulbo oculare ed organi accessori (cristallino, apparato oculomotore, palpebre, ghiandole lacrimali, congiuntiva).

Di seguito verranno analizzate sinteticamente le componenti coinvolte nel movimento dell’occhio, nella formazione dell’immagine e verrà descritto come sono innervate. Seguirà  poi un approfondimento sulle caratteristiche funzionali del sistema visivo.
\\\ \\\
\section{Anatomia e fisiologia}

\subsection{Bulbo oculare}

Il bulbo oculare si trova nella parte anteriore della cavità orbitaria decentrato in alto e verso l’esterno rispetto alle pareti e mantenuto in sito da connessioni muscolari, vascolari e dal nervo ottico. È inoltre appoggiato ad una formazione membranosa connettivale detta fascia del bulbo. 

Il bulbo oculare è una formazione sferoidale che contiene materiali liquidi, semi-liquidi e solidi fasciati da diverse membrane di copertura dette tonache. All’esterno si trova la tonaca fibrosa, formata dalla sclera e dalla cornea, poi la tonaca vascolare, ricchissima di vasi per irrorare e pigmentazione per permettere la creazione di una camera oscura nell’occhio, infine, all’interno, la tonaca nervosa (retina) che contiene i fotorecettori responsabili della trasformazione del segnale luminoso in segnale elettrico.

Il bulbo è diviso in due camere: la camera anteriore che contiene umor acqueo e la camera posteriore che contiene umor vitreo. Le camere sono separate dall’iride: un disco portante al centro il forame pupillare, il quale gestisce la quantità di luce da far entrare nell’occhio. Dietro l’iride si trova il cristallino, supportato dai processi ciliari, il quale è in grado di cambiare la sua forma perché avvenga il processo di accomodazione, cioè la messa a fuoco a diverse distanze.

\\\


\begin{figure}[h!]
\centering
\begin{minipage}{.5\textwidth}
  \centering
  \includegraphics[scale=1.0]{source/immagini/anatomia_occhio.jpg}
  \captionof{figure}{Sezione del bulbo oculare}
  \label{fig:test1}
\end{minipage}%
\begin{minipage}{.5\textwidth}
  \centering
  \includegraphics[scale=0.08]{source/immagini/retina.jpg}
  \captionof{figure}{Stratificazione della retina}
  \label{fig:test2}
\end{minipage}
\end{figure}


\subsection{Generazione dell'immagine retinica}

Quando la luce colpisce l’occhio deve attraversare la cornea, la pupilla e il cristallino, che permettono di focalizzarla su diverse parti della retina. La zona della retina in cui vengono codificati i dettagli e i colori è la fovea, dove si concentrano i fotorecettori coni. Perifericamente invece, sono presenti maggiormente i fotorecettori bastoncelli, i quali sono adibiti alla percezione della luminosità e del movimento. I coni e i bastoncelli hanno il compito di trasformare il segnale luminoso in segnale elettrico attraverso una serie di eventi biochimici a cascata. Il segnale viene poi trasmesso alle cellule bipolari e ganglionari. Di quest'ultime ne esistono diversi tipi che raccolgono le diverse caratteristiche dell’immagine visiva, come luminosità, colore o movimento. Gli assoni delle cellule ganglionari si uniscono per formare il nervo ottico, il quale giunge al chiasma ottico dove avviene una parziale decussazione delle fibre ottiche: quelle provenienti dalla metà nasale della retina passano nel tratto ottico del lato opposto, che contiene quindi fibre omolaterali della metà temporale e fibre controlaterali della metà nasale. Le fibre si dirigono poi al Nucleo Genicolato Laterale, al collicolo superiore o al pretetto. 
\\\
Vi sono diverse vie visive in base all’informazione da codificare: la via dalla retina al corpo genicolato laterale, la via retino-pretettale e la via retino-collicolare.
\\\ \\\ \\\ 
\textbf{Via retina-corpo genicolato laterale}: 
\\\
Al corpo genicolato laterale, nel talamo, arrivano le sole fibre adibite alla percezione visiva. Da qui si dipartono neuroni secondari che arrivano alla corteccia visiva primaria (V1), dalla quale si dipartono la via ventrale e la via dorsale che portano alle aree visive superiori, che sono specializzate funzionalmente. La via ventrale termina nel lobo temporale inferiore ed è adibita alla percezione dell’oggetto: colore, forma, trama, riconoscimento, visi. La via dorsale termina nel lobo parietale ed è adibita alla percezione spaziale, quindi la percezione delle differenze di luminosità, ma non di colore.
Le fibre che entrano a far parte delle vie riflesse, non si arrestano al corpo genicolato laterale, ma si portano al mesencefalo, terminando in due territori: pretetto e collicolo superiore.
\\\ \\\ \\\
\textbf{Via retino-pretettale}: 
\\\
Per il controllo dei riflessi pupillari le fibre delle cellule ganglionari adibite alla luminosità giungono al pretetto, zona formata da un gruppo di nuclei posti anteriormente e superiormente al collicolo superiore, nel punto in cui il mesencefalo si continua con il talamo. Le cellule di quest’area proiettano l’informazione ai nuclei di Edinger-Westphal, dal quale i neuroni raggiungono il muscolo sfintere dell’iride, unendo i loro assoni con il nervo oculomotore.

\begin{figure}[h!]
	\centering
	\includegraphics[scale=0.52]{source/immagini/tronco_encefalico.jpg}
	\caption{Tronco encefalico}{Tronco encefalico}
	\label{fig:test3}
\end{figure}
\\\ \\\ \\\
\textbf{Via retino-collicolare}: 
\\\
Per il controllo dei movimenti rapidi degli occhi le vie passano dal collicolo superiore, nel mesencefalo, diviso in sette strati in cui sono rappresentate le mappe sensoriali. Al collicolo giungono informazioni provenienti anche da altri sensi permettendogli di dirigere i movimenti oculari in direzione dello stimolo, indipendentemente dalla sua natura.
\end{itemize}


\begin{figure}[h!]
\centering
\begin{minipage}{.5\textwidth}
  \centering
  \includegraphics[scale=0.2]{source/immagini/via_ascendente.jpg}
  \captionof{figure}{Schema vie ottiche riflesse per le risposte pupillari di tipo ortosimpatico (midriasi) e per le risposte di tipo parasimpatico (accomodazione con miosi) conseguenti all'osservazione di oggetti vicini (riflesso di convergenza-accomodazione)}
  \label{fig:test4}
\end{minipage}%
\begin{minipage}{.5\textwidth}
  \centering
  \includegraphics[scale=0.09]{source/immagini/via_discendente.jpg}
  \captionof{figure}{Schema vie ottiche proiettive e riflesse (miosi)}
  \label{fig:test5}
\end{minipage}
\end{figure}

\subsection{Aparato locomotore}

L’apparato locomotore svolge diverse funzioni: amplia il campo visivo, consente la fissazione foveale e la visione binoculare singola, fornisce la consapevolezza spaziale e stabilisce l'immagine retinica.
\\\
 
Il movimento oculare è permesso dai muscoli estrinseci i quali partono dalla profondità dell’orbita dove si trova un anello fibroso detto anello tendineo di Zinn. Da esso partono sei muscoli: quattro retti (inferiore, mediale, laterale e superiore), il muscolo obliquo superiore ed il muscolo elevatore della palpebra. C’è un ulteriore muscolo, l’obliquo inferiore, che però parte più avanti dell’anello tendineo. Il compito di questi muscoli è quello di permettere il movimento del globo e di mantenerlo sospeso all’interno di un sistema di tessuti connettivi estesi tra l’apice dell’orbita e le rime palpebrali. I muscoli retti, con i loro annessi, costituiscono un cono muscolare attraversato dal nervo ottico, vasi sanguigni e nervi. Lo spazio rimanente è riempito dai grassi retro bulbari. 
\\\
\begin{figure}[h!]
	\centering
	\includegraphics[scale=2.7]{source/immagini/muscoli_estrinseci_occhio.jpg}
	\caption{Muscoli estrinseci dell'occhio}{Muscoli estrinseci dell'occhio}
	\label{fig:test6}
\end{figure}

Il bulbo oculare è in grado di svolgere rotazioni attorno a tre assi: orizzontale $Z$, verticale $X$ e antero-posteriore $Y$ detti \emph{Assi di Fick}. Gli assi $Y$ e $Z$ giacciono sul piano equatoriale detto \emph{Piano di Listing}. Se l’occhio non compie alcuna rotazione si dice che si trova in posizione primaria, se ruota intorno ad un asse del piano di Listing  è in posizione secondaria, se invece ruota contemporaneamente intorno ad $X$ e $Z$ è in posizione terziaria.

\begin{figure}[h!]
	\centering
	\includegraphics[scale=0.49]{source/immagini/Assi.jpg}
	\caption{Assi di Fick}{Assi di Fick}
	\label{fig:test7}
\end{figure}
\\\ \\\ \\\
L’occhio può compiere rotazioni intorno agli assi verticale e orizzontale dette duzioni:
 \begin{itemize}
 \itemsep-0.7em 
 \item[--]asse orizzontale $Z$: elevazione e abbassamento
 \item[--]asse verticale $X$: adduzione e abduzione.
 \item[--]asse antero-posteriore $Y$: extorsione (movimento che porta il polo superiore della cornea verso il lato temporale) e intorsione (movimento che porta il polo superiore della cornea verso il lato nasale)
 \end{itemize}
\\\ \\\
\textbf{Muscoli Retti orizzontali}
\\\
I muscoli retti laterale e mediale in contrazione determinano la rotazione dell’occhio attorno all’asse X, producendo Adduzione (retto mediale) e Abduzione (retto laterale).
\\\ \\\	
\textbf{Muscoli Retti verticali}
\\\
Con l’occhio in posizione primaria di sguardo i muscoli retti verticali hanno gli assi che formano 23$^{\circ}$ con l’asse visivo, quindi non ci sarà un movimento di pura elevazione o puro abbassamento, ma mista. Quando l’occhio è abdotto di 23$^{\circ}$ allora ci sarà movimento di pura elevazione quando si contrae il muscolo retto superiore e puro abbassamento quando si contrae il retto inferiore.
\\\ \\\
\textbf{Muscoli Obliqui}
\\\ 
In posizione primaria di sguardo l’asse muscolare forma un angolo di 53$^{\circ}$ con l’asse visivo, quindi in questa posizione la loro azione sarà molteplice. Obliquo superiore avrà come azione primaria l’incicloduzione, secondaria la depressione e terziaria l’abduzione. L’obliquo inferiore avrà azione primaria l’excicloduzione, secondaria l’elevazione e terziaria l’abduzione. 

\begin{figure}[h!]
\centering
\begin{minipage}{.5\textwidth}
  \centering
  \includegraphics[scale=0.25]{source/immagini/Assi_obliqui.png}
  \captionof{figure}{Asse del muscolo obliquo}
  \label{fig:test8}
\end{minipage}%
\begin{minipage}{.5\textwidth}
  \centering
  \includegraphics[scale=0.34]{source/immagini/Assi_retti.png}
  \captionof{figure}{Asse del muscolo retto}
  \label{fig:test9}
\end{minipage}
\end{figure} 

\subsection{Innervazione del sistema visivo}

L’innervazione del sistema visivo è costituita dai nervi encefalici. In generale ci sono 12 nervi encefalici, ma quelli adibiti al sistema visivo sono cinque: II paio o nervo ottico, III paio o nervo oculomotore comune, IV paio o nervo trocleare, V paio o nervo trigemino, VI paio o nervo abducente.
\\\ \\\
\textbf{Nervo ottico}
\\\ 
Esso prende origine dalla papilla del nervo ottico, costituito dall’unione degli assoni delle cellule gangliari della retina, abbandona il bulbo oculare e si incontra nel chiasma ottico con il nervo ottico dell'altro occhio per scambiarsi le fibre. Proseguono poi nei tratti ottici che giungono nei corpi genicolati laterali del diencefalo.
\\\ \\\
\textbf{Nervo oculomotore comune}: 
\\\ E' composto da fibre motrici somatiche che originano dai nuclei dell’oculomotore (mesencefalo) recandosi nei muscoli estrinseci dell’occhio, e da fibre effettrici 	viscerali che nascono dal nucleo di Edinger-Westphal (nel mesencefalo) e si recano a due muscoli intrinseci dell’occhio (muscolo sfintere della pupilla e muscolo ciliare). Quando giunge alla fessura orbitaria superiore il nervo si divide nel ramo superiore e ramo inferiore. Il ramo superiore si distribuisce al muscolo retto superiore e al muscolo elevatore della palpebra superiore. Il ramo inferiore innerva il muscolo retto mediale, retto inferiore e obliquo inferiore. Dal ramo del muscolo obliquo superiore, si stacca un ramo che raggiunge il ganglio ciliare.
\\\ \\\
\textbf{Nervo trocleare (patetico)}: 
\\\
E' un nervo motore somatico, costituito da fibre che prendono origine nel mesencefalo dal nucleo del nervo trocleare e si distribuiscono al muscolo obliquo superiore dell’occhio.
\\\ \\\
\textbf{Nervo abducente}: 
\\\ E'un nervo motore somatico, ha origine nel ponte, dal nucleo del nervo abducente e provvede ad innervare il muscolo retto laterale dell’occhio. Anche il nervo abducente trasporta fibre sensitive somatiche che raccolgono dal muscolo stimoli propriocettivi. Tali fibre sensitive raggiungono il nervo oftalmico.
\\\ \\\
\begin{figure}[h!]
	\centering
	\includegraphics[scale=0.15]{source/immagini/nervo_abducente.jpg}
	\caption{Schema dell'origine e distribuzione dei nervi oculomotore, trocleare, abducente.}{Schema dell'origine e distribuzione dei nervi oculomotore, trocleare, abducente.}
	\label{fig:test10}
\end{figure}
\\\ \\\ 
\textbf{Nervo trigemino}: 
\\\ Esso è costituito da tre branche: oftalmica, mascellare e mandibolare. Contiene un gran numero di fibre sensitive somatiche e un minor numero di fibre motrici somatiche. La componente sensitiva somatica ha origine dal ganglio semilunare da cui viene inviato un prolungamento periferico nelle tre branche. Grazie a questo le fibre raccolgono stimoli sensitivi estrocettivi della cute della faccia, delle mucose dell’occhio, della bocca, del naso e inoltre stimoli propriocettivi dai muscoli estrinseci dell’occhio, dai muscoli mimici e dagli alveoli dentali. La componente motrice somatica origina dal nucleo masticatorio del trigemino e si distribuisce ai muscoli masticatori, al muscolo del martello, muscolo tensore del velo del palato, muscolo miloioideo e al ventre anteriore del muscolo digastrico. 
Alle tre branche si trovano annessi diversi gangli parasimpatici: ganglio ciliare, ganglio sfenopalatino, gangli sottomandibolare e sottolinguale ed il ganglio ottico. 
Il \emph{nervo oftalmico} nel suo decorso trova il ganglio ciliare, dal quale partono le fibre parasimpatiche che si portano ai musco intrinseci del bulbo oculare. Prima di raggiungere la fessura orbitaria superiore, si divide in nervo nasociliare, nervo frontale e nervo lacrimale.
Il \emph{nervo mascellare} si distribuisce ad un’estesa area cutanea della faccia ed alla mucosa delle cavità nasali e della bocca. È annesso al ganglio sfenopalatino, centro intercalato sul decorso delle fibre parasimpatiche che innervano la ghiandola lacrimale e le ghiandole della mucosa nasale e del palato.
Il \emph{nervo mandibolare} è un nervo misto costituito da fibre motrici somatiche e sensitive somatiche. Innerva i muscoli masticatori con la componente che giunge dal nucleo motore del ponte; con la componente somatosensitiva che origina dal ganglio semilunare si distribuisce alla cute della parte inferiore della faccia ed a parte della mucosa buccale. 

\begin{figure}[h!]
	\centering
	\includegraphics[scale=0.7]{source/immagini/nervo_trigemino.jpg}
	\caption{Nervo trigemino.}{Nervo trigemino.}
	\label{fig:test11}
\end{figure}

\section{Funzioni}

\subsection{Stato refreattivo dell'occhio}

Il potere refrattivo dell’occhio può essere considerato la somma di una serie di superfici refrattive, quali la cornea, il cristallino, l’umor acqueo e l’umor vitreo, oppure come la somma delle distanze che separano ogni superficie con differente indice di rifrazione. Dal momento che il potere è determinato dai raggi di curvatura delle superfici e dal loro indice e siccome lo stato refrattivo finale è calcolato dalla relazione del potere refrattivo con la retina, l’ametropia, cioè l’anormale condizione refrattiva dell’occhio, è suddivisa in due categorie:
 \begin{description} 
 \item[assiale:]dipende dalla lunghezza dell’occhio, 
 \item[refrattiva:]dipende dal potere dell’occhio. Il cambiamento del potere dell’occhio può essere dovuto all’indice di rifrazione dei mezzi oppure alla curvatura degli stessi.
Un soggetto è considerato emmetrope quando, guardando all’infinito, il fuoco del sistema di lenti dell’occhio cade sulla retina e perciò l’immagine viene vista a fuoco. Nel caso avvenga una variazione della perfetta coincidenza del fuoco principale dell’occhio con la retina, allora si hanno anomalie rifrattive o errori di rifrazione.
\end{description}
Gli errori rifrattivi si classificano in base a dove cade il fuoco del sistema occhio rispetto alla retina:
 \begin{description}
 \item[miopia:]se il fuoco cade anteriormente alla retina, causato da un occhio con potere maggiore oppure con lunghezza assiale maggiore.
 \item[ipermetropia:]se il fuoco cade posteriormente alla retina, causato da un occhio con potere minore oppure un occhio con lunghezza assiale minore.
 \item[astigmatismo:]se non esiste un fuoco singolo per tutti i meridiani dell’occhio ed è detto miopico se i fuochi cadono prima della retina o ipermetropico se cadono dopo la retina.
\end{description}


\subsection{Accomodazione}

L’accomodazione è il meccanismo che consente di aumentare il potere convergente dell’occhio e quindi di vedere nitidamente gli oggetti a tutte le distanze. Quello che avviene è un incremento dello spessore del cristallino e una diminuzione del suo diametro equatoriale, maggiore convessità della superficie anteriore, aumento della curvatura della superficie posteriore, il polo anteriore si avvicina alla cornea, mentre quello posteriore si allontana.
L’accomodazione risulta rilassata in un soggetto emmetrope quando guarda all’infinito, mentre la attiva per guardare a diverse distanze ravvicinate. Usando il massimo potere accomodativo, si ha a fuoco una distanza prossimale che si definisce punto prossimo, ed è la distanza può vicina all’occhio vista ancora nitidamente. Il punto remoto è invece la maggiore distanza che può essere vista nitidamente. La differenza tra punto remoto e punto prossimo è detta estensione accomodativa, mentre la stessa differenza ma espressa in diottrie è detta AMPIEZZA DI ACCOMODATIVA. L’ampiezza accomodativa è la quantità massima di potere accomodativo o diottrico, che l’occhio può aggiungere partendo da una condizione di emmetropizzazione. Essa viene calcolata facendo il reciproco diottrico del punto prossimo, misurato in metri, con occhio emmetrope o emmetropizzato.
Con il passare degli anni la capacità di variazioni di messa a fuoco dell’occhio subisce una lenta e continua riduzione ed oltre un certo valore assume il nome di presbiopia. Questa è dovuta ad una diminuzione della capacità di modifica della curvatura delle superfici del cristallino, per perdita di elasticità della capsula.

La variazione accomodativa può essere considerata come una retroazione ad un particolare stimolo, ma può anche essere indotta indirettamente quando vi è un atto di convergenza volontario. Nell’osservazione di un oggetto a distanza prossimale sono diversi i fattori che influenzano l’accomodazione, tra questi: la sfocatura dell’immagine retinica, il movimento dell’oggetto, la sua distanza apparente, la binocularità. Si presume che l’accomodazione sia servita sia dal sistema nervoso simpatico che parasimpatico. Il sistema parasimpatico ha il ruolo principale e le sue fibre pregangliari originano dal nucleo di Edinger-Westphal, a livello del mesencefalo in prossimità del nucleo oculomotore, procede con il terzo nervo cranico, raggiungono poi il ganglio ciliare, da dove si dipartono fibre postgangliari che penetrano nell’occhio lungo i nervi ciliari corti, mentre altre fibre viaggiano con i nervi ciliari lunghi. Sul ruolo del sistema simpatico sono presenti delle controversie, ma sembra che i due sistemi abbiano ruolo antagonista, come in tutte le attività autonome, con effetto dominante del parasimpatico nella stimolazione accomodativa ed una funzione di aiuto nel passaggio dal vicino al lontano per il simpatico. Per la componente simpatica che innerva il muscolo ciliare, si sa che origina nel diencefalo e viaggia lungo il midollo spinale fino al più basso segmento cervicale e il più alto segmento toracico, fa sinapsi con il centro cilio-spinale di Budge, nel tratto intermediolaterale del midollo. Da qui i nervi di secondo ordine lasciano il midollo dall’ultima vertebra cervicale e prima toracica; queste fibre preganglionari scorrono e fanno sinapsi con il ganglio cervicale superiore. Le fibre continuano fino al plesso carotideo ed entrano nell’orbita, o con la prima divisione del nervo trigemino (oftalmico), oppure indipendentemente, dove possono unirsi con i nervi ciliari brevi e lunghi.

La stimolazione dell’accomodazione avviene nella seguente sequenza di eventi:
 \begin{enumerate}
  \itemsep-0.5em 
 \item i coni della retina sono stimolati dall’annebbiamento
 \item la somma dei segnali di annebbiamento viene trasmessa attraverso lo strato magnocellulare del nucleo genicolato laterale alla corteccia visiva
 \item la somma delle risposte delle cellule corticali formula il segnale di annebbiamento
 \item il segnale viene trasmesso anche all’area parieto-temporale e al cervelletto per processare l’informazione
 \item il segnale va al mesencefalo, ai nuclei oculomotori, ai nuclei di Edinger-Westphal dove viene formulato il comando
 \item il comando motore viene trasmesso al muscolo ciliare tramite il nervo oculomotore, il ganglio ciliare e poi i nervi ciliari corti
 \item viene cambiato lo stato di contrazione del muscolo ciliare
 \item il cristallino si deforma per ottenere la messa a fuoco sulla retina dell’immagine e quindi una visione più chiara.
\end{enumerate}

\begin{figure}[h!]
	\centering
	\includegraphics[scale=0.70]{source/immagini/accomodazione.png}
	\caption{Vie parasimpatica e simpatica del muscolo ciliare}{Vie parasimpatica e simpatica del muscolo ciliare}
	\label{fig:test12}
\end{figure}
 
Associata alla risposta accomodativa vi sono i fenomeni della contrazione pupillare e convergenza, che avvengono per sincinesia e costituiscono la reazione al punto prossimo. La risposta accomodativa avviene per un cambio di vergenza della luce a livello retinico in seguito all’avvicinamento dell’oggetto. La miosi è una risposta riflessa e la configurazione del nervo oculomotore suggerisce che le tre funzioni siano collegate. Però la convergenza è solo uno dei fattori che influenzano l’accomodazione, come lo stimolo di convergenza non deriva integralmente dall’accomodazione (Flom 1960).

Gli oggetti situati a diverse distanze nello spazio sono percepiti singoli grazie alla capacità degli occhi di mutare il loro allineamento nello spazio. Per ottenere la visione binoculare singola è necessario che le due fovee, che sono i punti retinici corrispondenti, siano allineate sullo stesso oggetto di interesse, per cui si dice che i due occhi convergono sul medesimo oggetto. Nell’utilizzo quotidiano degli occhi è necessario che la convergenza e l’accomodazione funzionino in armonia per consentire una nitida visione binoculare singola. Quando gli occhi convergono si ha la stimolazione indotta dell’accomodazione, chiamata accomodazione convergente. Se un occhio o gli occhi accomodano sopravviene una stimolazione della convergenza, che prende il nome di convergenza accomodativa. Convergenza e accomodazione, abbiamo detto essere collegate, ma con un certo grado di libertà. Infatti la convergenza continua ad essere funzionante anche in persone presbiti con ridottissima capacità accomodativa, mentre il giovane ipermetrope è abituato ad utilizzare una quantità di accomodazione in eccesso alla quantità utile di convergenza. L’ammontare di accomodazione utilizzabile senza mutamento della convergenza si chiama accomodazione relativa. Allo stesso modo la quantità di convergenza che può variare mantenendo l’accomodazione stabile, è detta convergenza relativa.



\subsection{Fusione e binocularità}

La visione singola, in condizioni binoculari, è permessa dal fatto che per ogni elemento retinico stimolato da uno dei due occhi esiste un corrispondente elemento retinico nell’altro occhio con la stessa localizzazione spaziale. L’elemento retinico è l’insieme delle strutture che elaborano la sensazione visiva in risposta alla stimolazione di un’area unitaria della retina. Ciascun elemento retinico localizza lo stimolo in una determinata direzione detta “direzione visiva della percezione”, e la linea che unisce il punto fissato con la fovea è la linea principale di direzione.
La visione binoculare avviene in due fasi. La prima, la visione monoculare simultanea, è dovuta alla stimolazione visiva dei due occhi. La seconda è la fusione delle due percezioni monoculari in una percezione singola a livello della corteccia occipitale.

Ciascun recettore retinico ha nell’altra retina un corrispondente recettore con la stessa localizzazione spaziale. Punti retinici con la stessa localizzazione spaziale sono detti “punti retinici corrispondenti”. La corrispondenza retinica spiega la visione binoculare singola, cioè la fusione sensoriale, che è un complesso meccanismo fisiologico attraverso il quale le immagini che colpiscono le due retine in punti corrispondenti, dopo essere giunte alla corteccia visiva, vengono percepite come un’unica immagine che rappresenta la fusione delle due immagini primitive. La superficie immaginaria nello spazio composta da tutti i punti le cui immagini cadono su punti retinici corrispondenti è detta oroptero. La fusione può avvenire anche in caso l’immagine non colpisca la retina in punti retinici perfettamente corrispondenti, l’importante è che si trovino all’interno dell’area di panum, l’area intorno all’oroptero in cui possono trovarsi gli oggetti per essere visti singoli. La disparità tra le immagini che vengono fuse entro i limiti dell’area di panum costituisce la base fisiologica per la percezione della stereopsi.

\\\ \\\
\textbf{Stereopsi}
\\\
La stereopsi è la capacità di fondere immagini leggermente differenti percepite da ciascun occhio durante l’osservazione di un oggetto tridimensionale. La fusione avviene se vengono stimolati punti retinici corrispondenti o con piccole disparità, in modo che non creino diplopia. La fusione può essere motoria o sensoriale, la prima porta a un aggiustamento della posizione dei due occhi per centrarsi sul punto di fissazione, mentre la seconda è un’integrazione mentale dei segnali provenienti dai due occhi. Per mantenere la fusione sensoriale è necessario che i due occhi siano sempre bene allineati e che le loro direzioni visive principali si incontrino nel punto di fissazione, se non dovesse succedere avverrebbe la diplopia.
Per osservare gli oggetti a diverse distanze si effettuano movimenti coordinati degli occhi che possono essere volontari o riflessi. Gli occhi possono compiere movimenti coniugati (versioni) dove ruotano nella stessa direzione e con la stessa ampiezza, oppure non coniugati (vergenze) dove si muovono in direzione opposta per spostare nello spazio il punto di intersezione degli assi visivi. Quando osserviamo un oggetto lontano e poi lo avviciniamo, avviene il fenomeno della convergenza. La convergenza può essere misurata in diottrie prismatiche, che è quella quantità di prisma in grado di determinare lo spostamento del raggio di luce di 1 cm alla distanza di 1 m. 
Come detto prima, l’accomodazione e la convergenza interagiscono. Può avvenire la risposta in convergenza prodotta da un’unità di stimolo accomodativo e può essere espressa tramite il rapporto convergenza accomodativa/accomodazione (AC/A), che misura la capacità di risposta della funzione di convergenza alla stimolazione di una unità di accomodazione. L’accomodazione e la convergenza sono controllate da due sistemi differenti: autonomo per l’accomodazione e volontario o scheletrico per la convergenza. Per avere una visione binoculare efficiente queste due componenti devono funzionare in armonia.
\\\ \\\
\textbf{Eteroforie}
\\\
La visione binoculare può essere mantenuta in presenza di una perfetta coordinazione dell’apparato neuromuscolare dei due occhi, però essendo un sistema complicato, raramente si ha questa condizione. Anche in assenza di una adeguata centratura, però, è possibile mantenere gli occhi allineati sull’oggetto grazie ai riflessi fusionali compensatori. Le condizioni che ne richiedono l’uso sono le “eteroforie o strabismo latente”, se il riflesso è assente, allora si ha “strabismo manifesto”, se si presenta saltuariamente allora si ha “strabismo intermittente o tropia”. Le forie possono essere di tre tipi in base alla direzione della deviazione:
 \begin{description}
 \item[ortoforia:]quando gli occhi sono centrati senza l’utilizzo di sforzi fusionali
 \item[exoforia:]quando, interrompendo la fusione con un mezzo dissociatore, gli occhi non rimangono allineati al punto di fissazione, ma uno devia verso l’esterno (tempia)
 \item[soforia:]quando gli occhi, in condizione di dissociazione, deviano verso l’interno (naso)
\end{description}
\\\ \\\
\textbf{Disparità di fissazione}
\\\
Si è detto che la fusione può avvenire per punti retinici disparati, a patto che si trovino nell’area di Panum. In caso di eteroforia le immagini possono non stimolare esattamente le due fovee, ma la disparità può essere lieve e gli occhi possono deviare solo di un’entità pari alla disparità di fissazione. La disparità di fissazione è una condizione presente in eteroforici con visione binoculare singola, rappresenta quindi l’ammontare di slittamento retinico consentito all’eteroforico nel limite della visione binoculare.